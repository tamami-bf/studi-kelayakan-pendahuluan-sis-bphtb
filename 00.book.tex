%%%%%%%%%%%%%%%%%%%% book.tex %%%%%%%%%%%%%%%%%%%%%%%%%%%%%
%
% sample root file for the chapters of your "monograph"
%
% Use this file as a template for your own input.
%
%%%%%%%%%%%%%%%% Springer-Verlag %%%%%%%%%%%%%%%%%%%%%%%%%%


% RECOMMENDED %%%%%%%%%%%%%%%%%%%%%%%%%%%%%%%%%%%%%%%%%%%%%%%%%%%
\documentclass[pdftex,12pt, oneside]{article}

% choose options for [] as required from the list
% in the Reference Guide, Sect. 2.2
%\usepackage[paperwidth=8.5in, paperheight=13in]{geometry} % Folio
\usepackage[paperwidth=8.27in, paperheight=11.69in]{geometry} % A4

\usepackage{makeidx}         % allows index generation
\usepackage{graphicx}        % standard LaTeX graphics tool
                             % when including figure files
%\usepackage{multicol}        % used for the two-column index
\usepackage[bottom]{footmisc}% places footnotes at page bottom
\usepackage[english]{babel}
\usepackage{enumerate}
\usepackage{paralist}
\usepackage{float}
\usepackage{gensymb}  
\usepackage{listings}
%\usepackage{siunitx}
% etc.
% see the list of further useful packages
% in the Reference Guide, Sects. 2.3, 3.1-3.3
\renewcommand{\baselinestretch}{1.5}

\newcommand{\HRule}{\rule{\linewidth}{0.5mm}}

%\makeindex             % used for the subject index
                       % please use the style svind.ist with
                       % your makeindex program


%%%%%%%%%%%%%%%%%%%%%%%%%%%%%%%%%%%%%%%%%%%%%%%%%%%%%%%%%%%%%%%%%%%%%

\begin{document}

%\author{Priyanto Tamami}
%\title{BUKU PETUNJUK OPERASIONAL SISTEM INFORMASI GEOGRAFIS UNTUK PBB-P2 DENGAN MAPINFO VERSI 8.0}
%\date{22 Desember 2015}
%\maketitle

%\input{./01.title.tex}
\begin{center}
{\large STUDI KELAYAKAN PENDAHULUAN PENGOLAHAN DATA SISTEM BEA PEROLEHAN HAK ATAS TANAH DAN/ATAU BANGUNAN}
\\[1cm]
14 Maret 2016\\
Priyanto Tamami, S.Kom.
\end{center}

%\frontmatter%%%%%%%%%%%%%%%%%%%%%%%%%%%%%%%%%%%%%%%%%%%%%%%%%%%%%%

%\include{dedic}
%\include{pref}

%\include{02.pengesahan} 

%\tableofcontents
%\listoffigures

%\mainmatter%%%%%%%%%%%%%%%%%%%%%%%%%%%%%%%%%%%%%%%%%%%%%%%%%%%%%%%
%\include{part}
%\include{chapter}
%\include{chapter}
%\appendix
%\include{appendix}

%\include{03.konsep-sig}
%\include{04.pengenalan-software}
%\include{05.koordinat}
%\include{06.registrasi-transformasi-koordinat} 
%\include{07.digitasi-on-screen} 
%\include{08.query} 

%\backmatter%%%%%%%%%%%%%%%%%%%%%%%%%%%%%%%%%%%%%%%%%%%%%%%%%%%%%%%
%\include{solutions}
%\include{referenc}
%\printindex

%%%%%%%%%%%%%%%%%%%%%%%%%%%%%%%%%%%%%%%%%%%%%%%%%%%%%%%%%%%%%%%%%%%%%%

\section{PENENTUAN MASALAH DAN PELUANG YANG DITUJU SISTEM}

\subsection{Penentuan Masalah}

Kondisi yang dilakukan sekarang dalam pencatatan administrasi Bea Perolehan Hak atas Tanah dan/atau Bangunan (BPHTB) dilakukan dengan menggunakan aplikasi Microsoft Office Excel.

Kondisi demikian menjadikan data tersebar pada komputer personal yang menangani pelayanan dan \textit{entry} data BPHTB sehingga data tidak dapat diakses bersama-sama, baik dalam \textit{entry} datanya, maupun dalam menyajikan informasi hasilnya, sekalipun dapat diakses bersama, maka dalam pengerjaannya akan muncul permasalahan teknis yang berkaitan dengan 

Beberapa hal yang menjadi masalah yang ditimbulkan dari data tersebar ini adalah apabila ada kerusakan pada personal komputer yang menangani pelayanan dan \textit{entry} data BPHTB, maka seluruh data harus dipindah ke personal komputer yang lain apabila memungkinkan, bila kerusakan yang terjadi ada pada \textit{harddisk} komputer yang bersangkutan, maka data yang tersimpan akan rusak.

Kondisi permasalahan lain yang ditemukan dalam pencatatan manual adalah data Objek Pajak untuk Pajak Bumi dan Bangunan Perdesaan dan Perkotaan (PBB-P2) berbentuk \textit{file} Microsoft Office Excel yang diekspor dari basis data SISMIOP yang ukurannya cukup besar, sehingga pada saat diakses dari aplikasi Microsoft Office Excel memerlukan waktu yang cukup lama.

\textit{File} Objek Pajak PBB-P2 yang berbentuk Microsoft Office Excel diperlukan guna menampilkan informasi Nilai Jual Objek Pajak (NJOP) yang nantinya akan menjadi nilai pembanding terhadap nilai transaksi atau nilai peralihan yang dilaporkan oleh masyarakat wajib pajak.

Data objek pajak PBB-P2 ini pun akan tidak terlihat perubahannya apabila data yang diminta telah dilakukan perubahan oleh SISMIOP karena pengambilan data NJOP dilakukan satu kali dalam satu tahun yaitu diawal tahun setelah fase penetapan PBB-P2 dilakukan.

\subsection{Peluang Yang Dituju}

Dari beberapa permasalahan yang timbul pada sistem pencatatan administrasi secara manual di atas, peluang yang ditawarkan sistem aplikasi web yang baru yaitu data nantinya akan disimpan di \textit{server}. Sehingga dengan aplikasi yang dibangun, data dapat diakses bersama bahkan dapat dilakukan \textit{entry} bersama.

Pada saat pengambilan data NJOP PBB-P2, dilakukan secara langsung pada basis data SISMIOP, sehingga perubahan-perubahan yang dilakukan pada saat itu, dapat langsung diterima sebagai nilai pembanding pada perolehan BPHTB. Satu keunggulan dan peluang dari data terpusat ini, nantinya dapat diperluas aksesnya untuk keperluan yang lain, misalkan membuat modul terpisah untuk kemudian dapat berkomunikasi dengan aplikasi web yang lain, atau mungkin aplikasi berbentuk \textit{mobile} yang sekarang dikenal dengan \textit{smartphone} dengan tujuan, salah satu contohnya adalah memberitahu masyarakat wajib pajak atau kuasanya bahwa berkas pelayanan yang pernah diajukan untuk diproses verifikasi BPHTB sudah dapat diambil, atau mungkin memiliki kendala lain seperti cek lapangan, atau kekurangan berkas sehingga berkas tertunda prosesnya.

Hal-hal tersebut tentu saja merupakan salah satu bentuk komunikasi antara Pemerintah Daerah dalam hal ini Dinas Pendapatan dan Pengelolaan Keuangan dengan masyarakat Wajib Pajak atau Kuasanya dalam pengurusan berkas BPHTB. Dan memungkinkan pula bahwa hasil dari modul tersebut dapat meningkatkan produktifitas baik dari Dinas Pendapatan dan Pengelolaan Keuangan dalam pengurusan berkas verifikasi BPHTB, maupun dari sisi masyarakat wajib pajak yang tidak perlu berulang kali mengunjungi tempat pelayanan hanya untuk memastikan apakah berkas yang pernah diajukan sudah selesai untuk diverifikasi atau belum.

Data pada \textit{harddisk} server pun diamankan dengan cara redundansi data yang dilakukan otomatis oleh \textit{controller array} yang sudah dimiliki server, sehingga apabila ada kerusakan di salah satu \textit{harddisk} maka \textit{harddisk} yang lain akan menangani simpanan datanya sampai \textit{harddisk} yang rusak diganti dengan yang baru.

\section{PEMBENTUKAN SASARAN SISTEM BARU SECARA KESELURUHAN}

Sasaran yang akan dituju oleh sistem yang baru dibuat nantinya adalah yang utama menggantikan sistem pencatatan manual yang tadinya dilakukan dengan Microsoft Office Excel, nantinya digantikan dengan menggunakan aplikasi khusus yang dibangun untuk melakukan pencatatan administrasi BPHTB.

Sistem yang baru harus mampu menyediakan data-data dari basis data SISMIOP secara \textit{real time} untuk keperluan data pembanding pada nilai peralihan BPHTB. 

Sistem yang baru harus dapat memproduksi hasil keluaran sesuai seperti apa yang dilakukan pada pencatatan administrasi BPHTB secara manual. Hal ini diperlukan agar tidak ada perbedaan secara hasil keluaran, tetapi dapat mempercepat proses pencatatan.

\section{PENGIDENTIFIKASIAN PARA PEMAKAI SISTEM}

Pengguna sistem yang baru tentu saja dari fungsi pelayanan BPHTB yang melakukan serah terima berkas dengan masyarakat wajib pajak atau kuasanya, terlebih bila ada petugas entry data pun dapat menggunakan sistem ini secara bersama.

\section{PEMBENTUKAN LINGKUP SISTEM}

Karena luasnya hal yang dapat ditangani oleh aplikasi web yang dibangun berdasarkan kebutuhan, maka lingkup sistem untuk membangun aplikasi ini akan dibatasi hanya pada melakukan pencatatan administrasi dan memproduksi berkas-berkas yang diperlukan pada proses verifikasi BPHTB.

\section{PENGUSULAN PERANGKAT LUNAK DAN PERANGKAT KERAS UNTUK SISTEM BARU}

Kondisi perangkat lunak untuk sistem yang baru akan menggunakan basis data Postgresql yang secara legalitas dapat diperoleh, dipergunakan, dan dikembangkan secara gratis sebagaimana tertuang dalam lisensinya.

Untuk web server akan menggunakan Apache Tomcat yang juga gratis untuk digunakan dan bebas dikembangkan sebagaimana tertuang dalam lisensinya.

Perangkat lunak lain yang digunakan adalah \textit{Integrated Development Environtment} (IDE) Eclipse, ini adalah alat yang digunakan untuk membangun aplikasi dan dapat digunakan secara gratis sebagaimana tertuang dalam lisensinya.

Diperlukan perangkat IDE karena aplikasi akan dibangun dalam lingkungan Java menggunakan framework ZKoss tipe Community Edition, seperti perangkat lunak yang lain, untuk edisi ini pun kita dapat menggunakannya secara gratis.

Sedangkan untuk perangkat keras dimana sistem baru akan dipasang dapat menggunakan server BILLER yang hanya digunakan untuk melakukan rekonsiliasi nilai penerimaan PBB-P2 secara harian, dan ini pun hanya digunakan 1 (satu) kali dalam sehari. Sehingga untuk menggunakan perangkat keras server ini sangat memungkinkan.

\section{PEMBUATAN ANALISIS UNTUK MEMBUAT ATAU MEMBELI APLIKASI}

Keputusan untuk membeli atau membuat tentu menjadi perhatian tersendiri. Beberapa hal yang perlu diperhatikan dalam membeli aplikasi tentu saja akan melihat faktor tenggang waktu, kualitas, dan layanan yang diberikan. 

Dari sisi tenggang waktu, apakah sistem yang dibangun dapat diselesaikan secepat hitungan beberapa bulan saja, atau beberapa minggu saja. Tentunya faktor tenggang waktu akan mempengaruhi hasil kualitas yang didapat, biasanya dengan waktu yang cepat, akan didapat kualitas aplikasi yang kurang, mulai dari \textit{bugs} yang cukup banyak, atau mungkin kesalahan yang cukup fatal bisa diproduksi.

Dari sisi layanan sebetulnya kearah layanan jasa, apakah kita akan menggunakan sistem kontrak putus, artinya begitu satu sistem sudah jadi, terpasang, dan siap digunakan pada fase produksi, maka ditunjuk 1 (satu) atau lebih orang untuk melakukan pemeliharaan dasar, atau bahkan diserahterimakan kode program untuk kemudian dapat dipeliharan dan dikembangkan mandiri secara internal, atau menggunakan sistem perpanjangan layanan jasa, seperti apabila dirasa ada yang kurang tepat, maka pihak ke-3 akan segera memperbaiki, dan selama kita menggunakan aplikasi tersebut, harus selalu mengeluarkan biaya secara rutin (yang biasanya setiap bulan) untuk melakukan pemeliharaan sistem.

Atau dengan pertimbangan lain adalah membuat aplikasi sendiri karena kita memiliki seorang fungional Pranata Komputer yang tugas pokok dan fungsinya memang dapat membangun suatu sistem aplikasi, tanpa mengeluarkan biaya tambahan yang biasanya cukup besar, namun tentunya waktu yang dibutuhkan lebih lama dari membeli aplikasi.

\section{PEMBUATAN ANALISIS BIAYA/MANFAAT}

Pengembangan sistem informasi seperti ini merupakan suatu investasi seperti halnya investasi proyek lainnya. Investasi berarti dikeluarkannya sumber-sumber daya untuk mendapatkan manfaat dimasa mendatang. Investasi untuk pengembangan sistem informasi juga membutuhkan sumber-sumber daya. Sebagai hasilnya, sistem informasi akan memberikan manfaat-manfaat yang dapat berupa penghematan-penghematan atau manfaat-manfaat yang baru.

Jika manfaat yang diharapkan lebih kecil dari sumber daya yang dikeluarkan, maka sistem informasi ini dikatakan tidak bernilai atau tidak layak.

\textbf{KOMPONEN BIAYA}

Ada dua komponen yang diperlukan untuk melakukan analisis biaya / efektivitas yakni :

\begin{enumerate}[1.]
  \item Komponen Biaya
  \item Komponen Efektivitas
\end{enumerate}

Biaya yang berhubungan dengan pengembangan sistem informasi dapat diklasifikasikan ke dalam 4 kategori utama, yaitu :

\begin{enumerate}[1.]
  \item Biaya pengadaan

Adalah semua biaya yang terjadi sehubungan dengan memperoleh perangkat keras. Yang termasuk biaya pengadaan sistem ini, karena seluruh perangkat keras menggunakan perangkat yang sudah ada sehingga biaya untuk pengadaan boleh dikatakan nihil.
  
  \item Biaya persiapan operasi
  
Adalah semua biaya untuk membuat sistem siap untuk dioperasikan. Untuk biaya persiapan operasi, karena menggunakan perangkat lunak yang berbasis \textit{free} dan \textit{open source}, kemudian dari sisi jaringan baik lokal maupun internet sudah terbangun sehingga biaya persiapan operasi pun bisa dikatakan nihil.

  \item Biaya proyek

Adalah semua biaya yang dikeluarkan untuk pengembangan sistem termasuk penerapannya. Yang termasuk diantaranya adalah biaya dalam tahap analisis sistem, biaya dalam tahap design sistem, dan biaya dalam tahap penerapan sistem. Tentunya bila menggunakan pihak lain untuk membangun aplikasi, akan muncul biaya-biaya tersebut, tetapi untuk dapat meningkatkan sisi manfaatnya, seorang fungsional pranata komputer pun dapat melakukan ketiga tahapan tersebut dengan tugas pokok dan fungsinya sebagai seorang fungsional pranata komputer, sehingga biaya proyek ini menjadi nihil.
  
  \item Biaya operasi dan biaya perawatan 
  
Adalah biaya-biaya yang dikeluarkan untuk mengoperasikan sistem supaya sistem dapat beroperasi. Karena menggunakan sistem perangkat keras yang sudah beroperasi, maka baik biaya-biaya pemakaian seperti listri, telepon, dan biaya rutin akan sama besarnya seperti sebelum menggunakan sistem aplikasi yang akan dibangun. Biaya-biaya personil pun melihat aplikasinya akan dibangun berdasarkan sistem alur bisnis yang sudah berjalan, maka tidak ada biaya tambahan yang lain. 

Dari sisi perawatan, baik perangkat lunak dan perangkat keras pun, cukup dilakukan oleh fungsional pranata komputer sesuai tugas pokok dan fungsinya. Dapat disimpulkan bahwa biaya operasi dan biaya perawatan tidak akan mengalami kenaikan seperti pada saat melakukan pencatatan administrasi secara manual.

\end{enumerate}

\textbf{KOMPONEN MANFAAT}

Manfaat yang didapat dari sistem informasi yang akan dibangun dapat diklasifikasi sebagai berikut :

\begin{enumerate}[1.]
  \item Manfaat mengurangi kesalahan-kesalahan
  
Dilihat dari sistem manual yang melakukan entry data secara manual, maka dimungkinkan akan terdapat kesalahan pada saat entry karena tidak adanya kontrol verifikasi. Dengan dibangunnya sistem ini, nantinya tiap bagian entry data akan dikontrol dan langsung diambilkan dari sumber datanya secara \textit{real time}

  \item Manfaat meningkatkan kecepatan aktivitas
  
Dari beberapa formulir yang diperlukan untuk \textit{entry} data yang terdiri dari beberapa lembar berkas, maka dengan dibangunnya sistem ini, data-data yang redundan (yang sama) tidak perlu dimasukkan beberapa kali, bahkan dengan memasukkan informasi inti tentang objek pajak seperti Nomor Objek Pajak (NOP) seluruh data yang berkaitan dengan objek pajak dapat diambilkan langsung oleh sistem dari basis data SISMIOP.

  \item Manfaat meningkatkan perencanaan dan pengendalian manajemen.
  
Dengan beberapa modul tambahan dimungkinkan untuk memberikan laporan-laporan statistik sebagai bahan untuk pengambilan keputusan, seperti berapa realisasi yang sudah tercapai sampai dengan hari ini, atau berapa jumlah berkas pelayanan yang tertunda, apa saja kendalanya sehingga dapat diselesaikan dengan cepat.

\end{enumerate}

\section{PENGKAJIAN TERHADAP RESIKO PROYEK}

Tentu saja untuk setiap proyek yang berjalan pasti memiliki resiko, yang apabila dapat ditangani dengan benar, bisa menjadi nilai tambah tersendiri terhadap proyek yang akan berlangsung.

Dalam kondisi pembangunan sistem pelayanan BPHTB ini resiko operasional yang mungkin kerap kita jumpai, yaitu adanya perpindahan fungsi petugas yang melakukan pelayanan, sehingga nantinya perlu dibentuk buku panduan untuk mengoperasionalkan aplikasi yang dimaksud.

Resiko yang lain yang mungkin timbul adalah Hazard Risk, atau resiko terkait kecelakaan fisik seperti kebakaran dan gempa bumi, yang solusinya adalah seluruh kode sistem dan desain proyek aplikasi sistem web ini disimpan di server luar (github.com) sehingga kode aman untuk dilakukan modifikasi atau inisialisasi ulang.

\section{PEMBERIAN REKOMENDASI UNTUK MENERUSKAN ATAU MENGHENTIKAN PROYEK}

Rekomendasi yang dapat diusulkan dengan melihat beberapa pertimbangan diatas yaitu meneruskan kegiatan proyek yang dikerjakan mandiri atau dibuat sendiri oleh seorang pranata komputer. Selain akan meningkatkan manfaat dari aplikasi yang dibuat 

\end{document}